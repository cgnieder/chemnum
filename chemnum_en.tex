\documentclass[load-preamble+,babel-options={english,american}]{cnltx-doc}
\usepackage[version=1]{chemnum}
\setcnltx{
  package = chemnum ,
  info    = \chemnum\ revisited ,
  authors = Clemens Niederberger ,
  add-cmds = {
    cmpd ,
    initcmpd ,
    labelcmpd ,
    cmpdplain ,
    cmpdproperty ,
    cmpdref ,
    replacecmpd ,
    resetcmpd ,
    setchemnum ,
    cmpdshowdef ,
    cmpdshowref ,
    newcmpdlabelformat ,
    subcmpdplain ,
    submaincmpdplain ,
    subcmpdproperty ,
    subcmpdshowdef ,
    subcmpdshowref ,
  } ,
  add-silent-cmds = {
    ch ,
    chemfig , chemname ,
    CNlabel ,
    CNlabelnoref ,
    CNlabelsub ,
    CNlabelsubnoref ,
    CNref ,
    CNrefsub ,
    CNsubnoref ,
    compound ,
    cs ,
    declarecompound ,
    detokenize,
    expandfull,expandtwice,
    fcite ,
    setatomsep ,
    theffbibliography ,
  } ,
  index-setup = { othercode=\footnotesize,level=\addsec },
  makeindex-setup = { columns=3,columnsep=1em }
}
\usepackage[artemisia]{textgreek}
\activatechemgreekmapping{textgreek}

\usepackage{chemformula}
\usepackage{chemfig}

\usepackage{array,booktabs}

\defbibheading{bibliography}[References]{\addsec{#1}}

\newcommand*\PDF{\cnltxacronym{PDF}{pdf}}
\newcommand*\EPS{\cnltxacronym{EPS}{eps}}
\newcommand*\PS{\cnltxacronym{PS}{ps}}
\newcommand*\ID{\cnltxacronym{ID}{id}}

\begin{document}

\section{License and Requirements}\label{sec:license-requirements}
\license

\chemnum\ requires the bundles \bnd{l3kernel}~\cite{bnd:l3kernel} and
\bnd{l3packages}~\cite{bnd:l3packages}.  It also requires the
\pkg{translations} package~\cite{pkg:translations} and \pkg{chemgreek} from
the \bnd{chemmacros} bundle~\cite{bnd:chemmacros}.

\section{News}\label{sec:news}
The \chemnum\ package has been my first attempt to create a comprehensive
labeling package for chemical compounds.  However, it had and has more than
one weakness and its code was -- to be frank -- a mess.  Version~1 is now a
complete re-write of \chemnum\ where I tried to achieve several points:
\begin{itemize}
  \item A cleaner code internally.
  \item A cleaner user interface, \ie, more user macros for different tasks, a
    unified naming of the commands and a less redundant naming of the
    options.
  \item Extended functionality such as sorting and compressing of sublabel
    lists.
\end{itemize}

Although the syntax is more or less the same some changes have been made that
make version~1 incompatible with version~0.  This is way Version~0 is still
available through the option \keyis{version}{0}.

Many commands have got a new name! The most important ones are:
\begin{itemize}
  \item \cs*{cmpdref}; this is now called \cs{replacecmpd}.
  \item \cs*{cmpdinit}; this is now called \cs{initcmpd}.
  \item \cs*{cmpdsetup}; this is now called \cs{setchemnum}.
\end{itemize}

\section{Background}\label{sec:background}
As far as I know there are three packages that are meant to help with
numbering chemical compounds.  All of them have their weaknesses.

The first one -- \pkg{chemcono}~\cite{pkg:chemcono} -- redefined bibliography
commands for that purpose.  Compounds have to be declared in what is called
\cs*{theffbibliography}.  Then one can reference them with \cs*{fcite}.
However, it produces a list of compounds in the text.  So the package author
suggests:
\begin{cnltxquote}[Stefan Schulz]
  After compilation and printout, discard the last page.
\end{cnltxquote}
Obviously that's not a perfect solution.

The second one -- \pkg{chemcompounds}~\cite{pkg:chemcompounds} -- was written,
because the author didn't want to work with the weaknesses of \pkg{chemcono}
any more.  When he wrote the package he basically used the same mechanism to
create the labels as \pkg{chemcono} did.
\begin{cnltxquote}[Stephan Schenk]
  When taking a closer look at the chemcono package, I realised that the only
  thing one has to do is to get rid of everything which produces text.  Thus,
  as a basis I used the mechanism of \cs*{bibitem} and \cs*{cite} in pretty
  much the same way as \pkg{chemcono} does by extracting the corresponding
  code from \code{article.cls} and \code{latex.ltx} but deleting any
  unnecessary commands producing output.  I also introduced several lines of
  code to make the printing of the compound names more customisable.
\end{cnltxquote}
Some points still left me unsatisfied, though:
\begin{enumerate}
  \item Compounds usually need to be declared with
    \cs*{declarecompound}.  They need to be declared in any case if you need a
    label like \cmpd{bsp.one}.  Then, one even needs to choose the label by
    hand, what somehow undermines the automatic numbering principle.
  \item The layout can't be changed for a single label but only for all.
  \item The numbers can't be reset.  \emph{Although in most cases this is
      neither necessary nor can it be recommended}, there can be individual
    cases where this would be useful.
  \item A list of several compounds \cs*{compound}\Marg{a,b,c} can only be
    customized with more effort than what would be convenient.
\end{enumerate}

Then there is \pkg{bpchem}~\cite{pkg:bpchem}, which provides commands similar
to \cs*{label} \cs*{ref}: \cs*{CNlabel}\marg{}, \cs*{CNlabelnoref}\marg{} and
\cs*{CNref}\marg{}.  It provides commands for sublabels, too:
\cs*{CNlabelsub}\marg{}\marg{}, \cs*{CNlabelsubnoref}\marg{}\marg{} and
\cs*{CNrefsub}\marg{}\marg{}.  This makes it more flexible than the others
regarding sublabels.  However, it barely provides possibilities to customize
the labels, lists are not possible and the fact that there are different
commands for labels and sublabels isn't the best solution, either.

\chemnum\ is intended to fill these gaps.  For this all commands have been
written from scratch. Some of the ideas of \pkg{chemcompounds} \eg\ regarding
delimiters and layout have been picked up, though.

If you notice any feature missing, please let me know by sending me an email.

\section{Overview over the Available Commands}\label{sec:overv-over-avail}

\begin{commands}
  \command{cmpd}[\sarg\code{+}\oarg{options}\marg{list of \ID s}]
    The main command for creating and refering to compound labels.  This
    command is described in detail in section~\ref{sec:deta-comp-labels}.
  \command{refcmpd}[\oarg{options}\marg{\ID}]
    This command only refers to a already defined label but does not define a
    label itself.  This is an alias of \cs{cmpd}\code{+}.
  \command{labelcmpd}[\oarg{options}\marg{\ID}]
    This command only defines a new label but does not print it.  This is an
    alias of \cs{cmpd}\sarg.
  \expandable\command{cmpdplain}[\marg{\ID}]
    Reads a label and writes it expandably without formatting.  It is not able
    to parse a list.  Its sole purpose is usage in pdfstrings
    (\cf\ \cs*{texorpdfstring}\marg{\TeX}\marg{pdfstring}).  This command is
    described in section~\ref{sec:deta-comp-labels}.
  \expandable\command{subcmpdplain}[\marg{main \ID}\marg{sub \ID}]
    Reads a sublabel and writes it expandably without formatting.  It is not
    able to parse a list.  Its sole purpose is usage in pdfstrings
    (\cf\ \cs*{texorpdfstring}\marg{\TeX}\marg{pdfstring}).  This command is
    described in section~\ref{sec:deta-comp-labels}.
  \expandable\command{submaincmpdplain}[\marg{main \ID}\marg{sub \ID}]
    Reads a main and a sublabel and writes them expandably without formatting.
    It is not able to parse a list.  Its sole purpose is usage in pdfstrings
    (\cf\ \cs*{texorpdfstring}\marg{\TeX}\marg{pdfstring}).  This command is
    described in section~\ref{sec:deta-comp-labels}.
  \command{initcmpd}[\oarg{options}\marg{list of \ID s}]
    Initiate compound labels.  This command can only be used in the preamble.
    It is desribed in section~\ref{sec:deta-comp-labels}.
  \expandable\command{cmpdproperty}[\marg{\ID}\marg{property}]
    Get the associated property \meta{property} of compound \meta{\ID}. This
    command is described in detail in section~\ref{sec:deta-comp-labels}.
  \expandable\command{subcmpdproperty}[\marg{main \ID}\marg{sub \ID}\marg{property}]
    Get the associated property \meta{property} of subcompound \meta{sub \ID}
    of compound \meta{main \ID}.  This command is described in detail in
    section~\ref{sec:deta-comp-labels}.
  \command{newcmpdlabelformat}[\marg{name}\marg{command}]
    Makes the label format \meta{name} known to \chemnum.  \meta{command}
    needs to be a command that takes an integer number as argument and should
    return a formatted version of it.  In practice you should not need to use
    this command as the most common formats already are defined.  This command
    is described in section~\ref{sec:counter-settings}.
  \command{resetcmpd}[\oarg{integer}]\Default{1}
    Reset the numbering for main compound labels to start with \meta{integer}
    again.  This is the same as
    \cs*{setcounter}\Marg{cmpdmain}\Marg{$\text{\meta{integer}}-1$}.  The
    command is described in section~\ref{sec:counter-settings}.
  \command{cmpdshowdef}[\marg{\ID}]
    Internal command used to display \meta{\ID} of a newly defined compound
    label when the option \option{show-keys} is used.  The command is
    described in section~\ref{sec:debugg-inform}.
  \command{cmpdshowref}[\marg{\ID}]
    Internal command used to display \meta{\ID} of a referencing compound label
    when the option \option{show-keys} is used.  The command is described in
    section~\ref{sec:debugg-inform}.
  \command{subcmpdshowdef}[\marg{main \ID}\marg{sub \ID}]
    Internal command used to display \meta{main \ID} and \meta{sub \ID} of a
    newly defined subcompound label when the option \option{show-keys} is
    used.  The command is described in section~\ref{sec:debugg-inform}.
  \command{subcmpdshowref}[\marg{main \ID}\marg{sub \ID}]
    Internal command used to display \meta{main \ID} and \meta{sub \ID} of a
    referencing subcompound label when the option \option{show-keys} is used.
    The command is described in section~\ref{sec:debugg-inform}.
\end{commands}

\section{Numbering Compounds}\label{sec:numbering-compounds}
\subsection{Main command}\label{sec:main-command}\resetcmpd

The main command of this package is this one:
\begin{commands}
  \command{cmpd}[\marg{\ID}]
    When \meta{compound name} is used the first time, the label is created,
    saved (= declared) and printed.  Each further use just prints the label.
\end{commands}

\begin{example}
  Compounds \cmpd{a} and \cmpd{b} are declared and can be used any time:
  \cmpd{a}.  No pre-declaring is necessary.  Compounds like \cmpd{c} are
  numbered in the order they appear in the text.\par
  Once again: \cmpd{b}, \cmpd{a}, \cmpd{c}.
\end{example}

If it is necessary to declare a compound without printing the label it is
possible with
\begin{commands}
  \command{cmpd}[\sarg\marg{\ID}]
    Declare the label but don't print anything.
\end{commands}

\begin{example}
  The hidden version\cmpd*{d} declares the label but doesn't print anything.
  The next \cmpd{e} continues to count with the next number.  With \cmpd{d}
  the label can be used, of course.
\end{example}

You can pretty much use what you like for a label name except for the
separator symbols (see section~\ref{sec:chang-input-mark}).  Be careful with
blanks though!  Leading and trailing spaces are ignored, spaces at other
places are not.  It's probably best not to use blanks in label names at all.

\begin{example}[sourcecode-options={showspaces=true}]
  \cmpd{aa}, \cmpd{aa }, \cmpd{ aa}, and \cmpd{ aa } all have the same label.
  Likewise \cmpd{a a}, \cmpd{a a }, \cmpd{ a a}, \cmpd{ a a }, \cmpd{a  a},
  \cmpd{a  a }, \cmpd{ a  a}, and \cmpd{ a  a }.
\end{example}

\subsection{Sublabel}\label{sec:sublabel}
If you want a label like \cmpd{a.one}, you need to use the following syntax:
\begin{commands}
  \command{cmpd}[\Marg{\meta{main \ID}.\meta{sub \ID}}]
    \meta{main \ID} is the main name which stays the same, \meta{sub \ID}
    varies.  This syntax means that the point \code{.} \emph{cannot} be a part
    of \meta{main \ID} or \meta{sub \ID}.  Instead of the point you also can
    use another symbol, see section~\ref{sec:chang-input-mark}.
\end{commands}

\begin{example}
  \cmpd{f.one} and \cmpd{f.two} are related, as are \cmpd{g.one} and
  \cmpd{g.two}.  Of course these labels can be used again: \cmpd{g.two} and
  \cmpd{f.one}.
\end{example}

This also works if the main name has already been used.
\begin{example}
  \cmpd{a} and its variants \cmpd{a.one} and \cmpd{a.two}
\end{example}

The same way the main name of combined labels can be used solely.
\begin{example}[side-by-side]
  \cmpd{f} and \cmpd{g}
\end{example}

How you can create a label like \cmpd{f.{one,two}} is explained in
section~\ref{sec:lists-rang-subl}.

\subsection{Lists}\label{sec:lists}
There is actually more to the \cs{cmpd} command.  It also prints lists of
labels.  The right description would be something like:
\begin{commands}
  \command{cmpd}[\marg{(possibly comma separated list of) label name(s)}]
    Treats each entry of the list as described before.
\end{commands}
This means that with default settings the comma can't be part of the label
name unless hidden in braces.  As separator can be used another symbol, too,
see section~\ref{sec:chang-input-mark}.

\begin{example}
  More than one label can be put inside \cs{cmpd}, separated by commas.  Then
  a list like \cmpd{a, b, c, e, g.two} is printed.
\end{example}
The Harvard comma (see section~\ref{sec:lang-depend-sett}) in \code{, and}
between \cmpd{e} and \cmpd{g.two} suggests that there are options to customize
the list, see section~\ref{sec:formatting-labels}.

\subsection{Lists and ranges of sublabels}\label{sec:lists-rang-subl}
Sometimes it can be useful to display a label with a list or a range of
sublabels.  Suppose you have compounds
\cmpd{q.one,q.two,q.three,q.four,q.five} which for example differ in their
substituents.  It can be useful to refer to them all at once:
% \cmpd[cmpd-all]{q}. TODO

The syntax is rather intuitive -- you just input a list of sublabels:
\begin{example}
  \setchemnum{compress=false}%
  list of labels: \cmpd{q.one, q.two, q.three, q.four, q.five}\par
  label with list of sublabels: \cmpd{q.{one,two,three,four,five}}
\end{example}
Since the sublist is input with a comma in the default setting, you have to
put them into braces.  If you add a list of sublabels to a main label they
will always be printed in the order the sublabels have been declared and not
in the order they're input in the list:

\begin{example}
  \setchemnum{compress=false}%
  compare \cmpd{q.{one,two,three,four,five}}
  with \cmpd{q.{five,four,three,two,one}} and
  \cmpd{q.{three,four,one,five,two}}
\end{example}

Using this syntax you also can create ranges of sublabels.  For this you
enable the option \option{compress}.  Or rather: this is the default setting.
\begin{example}[side-by-side]
  \cmpd{q.{two,four,three}} \par
  \cmpd{q.{five,one,three,four}} \par
  \cmpd{q.{one,three,five,two}}
\end{example}

Obviously you can't use a comma as part of a sublabel name.  You can change
the input marker, though.  See section~\ref{sec:overv-over-avail-1} for
available options.

\begin{example}
  % uses packages `chemfig', `chemformula' and `booktabs'
  \setatomsep{2em}%
  \chemname{\chemfig{*6(=-=-(-R)=-)}}{\cmpd{benzene.{H,Me,OH,NH2}}}
  \begin{tabular}{lll}
                                 & \ch{-R}   & Name \\\midrule
    \cmpd[sub-only]{benzene.H}   & \ch{-H}   & Benzene \\
    \cmpd[sub-only]{benzene.Me}  & \ch{-CH3} & Toluene \\
    \cmpd[sub-only]{benzene.OH}  & \ch{-OH}  & Phenol \\
    \cmpd[sub-only]{benzene.NH2} & \ch{-NH2} & Phenylamine (Aniline)
  \end{tabular}
\end{example}

\section{Details on Compound Labels}\label{sec:deta-comp-labels}
\subsection{How things work}\label{sec:how-things-work}

When you call \cs{cmpd} with a new label three things happen:
\begin{itemize}
  \item The new label gets initiated.  This is nothing more than adding it to
    an internal list.  The purpose of this is explained in
    section~\ref{sec:initiating-labels}.
  \item The new label gets declared.  This means that a number of internal
    commands are defined.  Amongst other things they hold a number of
    properties associated with the corresponding label.  Those properties are
    explained in more detail in section~\ref{sec:prop-comp-labels}.  The
    necessary information of the label are also written to the \code{aux}
    file.
  \item The label gets printed.
\end{itemize}

Since new labels are declared when \cs{cmpd} is first used using it in section
titles that are written to the table of contents may to lead to wrong
numbering.  In order to avoid this compound label information is written to
the \code{aux} file.  The command \cs{refcmpd}\oarg{options}\marg{\ID} only
reads those information but does not declare a label.  There is also a command
which does the opposite: it declares a label if it hasn't been declared before
but will not print the corresponding label:
\cs{labelcmpd}\oarg{options}\marg{\ID}.

Both commands have shortcut versions: \cs{cmpd}\code{+} is the same as
\cs{refcmpd}, \cs{cmpd}\sarg\ is the same as \cs{labelcmpd}.

Another command available is \cs{cmpdplain}\marg{\ID}.  This command is
similar to \cs{refcmpd}.  There are a few important differences, though:
\cs{cmpdplain} does \emph{not} take a list of labels as argument.  It also is
\emph{not} able to interpret sublabels.  \cs{cmpdplain} does not format the
label with whatever format has been declared.  And last but not least: it is
expandable.  This means it can be used to get labels in \PDF\ bookmarks.  It's
equivalent \cs{subcmpdplain}\marg{main \ID}\marg{sub \ID} does the same for
sublabels.  A third sibling, \cs{submaincmpdplain}\marg{main \ID}\marg{sub
  \ID}, writes both the main and the sublabel.

\subsection{Properties of compound labels}\label{sec:prop-comp-labels}

Every label has a number of properties.  The first property is of course its
\ID\ which identifies the label.  The other properties are:
\begin{description}
  \item[number] An internal unique number.
  \item[counter-representation] The counter representation associated with the
    label.  This is the actual label that get's printed.
  \item[pre-label-code] Code to be inserted before the label is printed.
  \item[post-label-code] Code to be inserted after the label is printed.
  \item[label-format] Formatting commands for the label.  This is most likely
    something like \cs*{bfseries}.
\end{description}

\begin{commands}
  \expandable\command{cmpdproperty}[\marg{\ID}\marg{property}]
    Get the associated property \meta{property} of compound \meta{\ID}.  This
    command is expandable.
\end{commands}

\begin{example}
  \newcommand*\expandfull{\romannumeral-`0}%
  \newcommand*\expandtwice{\detokenize\expandafter\expandafter\expandafter}%
  \ttfamily
  number: \cmpdproperty{benzene}{number}\par
  counter-representation: \cmpdproperty{benzene}{counter-representation}\par
  pre-label-code: \cmpdproperty{benzene}{pre-label-code}\par % empty
  post-label-code: \cmpdproperty{benzene}{post-label-code}\par % empty
  label-format: \expandtwice{\expandfull\cmpdproperty{benzene}{label-format}}
\end{example}

Similarly a sublabel has associated properties.  Additionally to the obvious
ones -- its \ID\ and the \ID\ the main label it belongs to -- these are
\begin{description}
  \item[number] An internal unique number.
  \item[counter-representation] The counter representation associated with the
    label.  This is the actual label that get's printed.
\end{description}

\begin{commands}
  \expandable\command{subcmpdproperty}[\marg{main \ID}\marg{sub \ID}\marg{property}]
    Get the associated property \meta{property} of subcompound \meta{sub \ID}
    of compound \meta{main \ID}.  This command is expandable.
\end{commands}

\begin{example}
  \ttfamily
  main-compound: \subcmpdproperty{benzene}{OH}{main-compound}\par
  number: \subcmpdproperty{benzene}{OH}{number}\par
  counter-representation: \subcmpdproperty{benzene}{OH}{counter-representation}
\end{example}

\subsection{Initiating labels}\label{sec:initiating-labels}
Initiating labels is not the same as declaring them.  In fact this may only be
useful in order to keep track of defined labels.  If you set the option

\section{Overview over the Available Options}\label{sec:overv-over-avail-1}
Except for the \option{version} option all of the following options are either
set as options to \cs{cmpd} or \cs{initcmpd} directly or via
\cs{setchemnum}\marg{options}, each time as a comma separated list of key/value
pairs.  Options that can be set via \cs{setchemnum} are marked with
\module{global}, those that only have an effect when used with \cs{cmpd} and
friends are marked with \module{compound}.  Those marked with \module{all} can
be set either way.

A few of the options only have an effect when used with the \cs{replacecmpd}
command.  They are marked with \module{replace}.

\begin{options}
  \keychoice{version}{0,1}\Default{1}
    Choose the package version.  This option can only be set as a package
    option!
  \keyval{counter-within}{counter}\Module{global}
    Reset the compound numbers when \meta{counter} is stepped.
  \keychoice{counter-format}{arabic,alph,Alph,roman,Roman,greek,Greek}%
    \Module{all}\Default{arabic}
    The format of the number associated with the main compounds.
  \keychoice{sub-counter-format}{arabic,alph,Alph,roman,Roman,greek,Greek}%
    \Module{all}\Default{alph}
    The format of the number associated with the sub compounds.
  \keybool{compress}\Module{all}\Default{false}
    If set to true a list of sublabels is compressed, \ie,
    \cmpd[compress=false]{q.{one,three,four,five}} becomes
    \cmpd{q.{one,three,four,five}}.
  \keyval{pre-label-code}{code}\Module{compound}\Default
    Code to be inserted before a label.
  \keyval{post-label-code}{code}\Module{compound}\Default
    Code to be inserted after a label.
  \keyval{main-sub-sep}{code}\Module{all}\Default{.}
    The separator symbol that is used in \cs{cmpd} to separate the \meta{main
      \ID} from a \meta{sub \ID}.
  \keyval{list-label-sep}{code}\Module{all}\Default{,}
    The separator that is used to separate different \meta{main \ID}s in
    \cs{cmpd}.
  \keyval{format}{formatting commands}\Module{all}\Default{\cs*{bfseries}}
    The default format of the labels.
  \keyval{list-sep-two}{code}%
    \Module{all}\Default{\visualizespaces{\GetTranslation{chemnum-sep-two}}}
    The output separator between labels in a list that contains of two items.
  \keyval{list-sep-more}{code}\Module{all}\Default{\visualizespaces{, }}
    The output separator between labels in a list that contains of more than
    two items.
  \keyval{list-sep-last-two}{code}%
    \Module{all}\Default{\visualizespaces{\GetTranslation{chemnum-sep-last-two}}}
    The output separator between the last two labels in a list that contains
    of more than two items.
  \keybool{sub-only}\Module{compound}\Default{false}
    If true the command \cs{cmpd} will only print sublabels but no main
    labels.
  \keybool{sub-all}\Module{compound}\Default{false}
    If true the command \cs{cmpd} will print all sublabels belonging to the
    corresponding main label.
  \keyval{sub-list-sep-two}{code}\Module{all}\Default{,}
    The output separator between labels in a sublist that contains of two
    items.
  \keyval{sub-list-sep-more}{code}\Module{all}\Default{,}
    The output separator between labels in a sublist that contains of more
    than two items.
  \keyval{sub-list-sep-last-two}{code}\Module{all}\Default{,}
    The output separator between the last two labels in a sublist that
    contains of more than two items.
  \keyval{sub-list-sep-range}{code}\Module{all}\Default{--}
    The output separator between two labels in a sublist denoting a range.
    This is only used when the option \option{compress} is active.
  \keybool{replace-auto}\Module{global}\Default{true}
    When set to true this adds an incremented integer to the replacement tag.
  \keyval{replace-tag}{text}\Module{global}\Default{TMP}
    The default replacement tag.
  \keyval{tag}{text}\Module{replace}\Default{TMP}
    The local replacement tag.
  \keyval{replace-style}{code}\Module{global}\Default{\cs*{sffamily}}
    Additional \TeX\ code that it placed before the \cs{cmpd} command in the
    replacement.
  \keyval{style}{code}\Module{replace}\Default{\cs*{sffamily}}
    Local additional \TeX\ code that it placed before the \cs{cmpd} command in
    the replacement.
  \keyval-{replace-pos}{\marg{\TeX\ pos}\marg{\PS\
      pos}}\Module{global}\Default{bb}
    Options for \pkg{psfrag}'s \cs{psfrag}.
  \keyval-{pos}{\marg{\TeX\ pos}\marg{\PS\
      pos}}\Module{replace}\Default{bb}
    Local options for \pkg{psfrag}'s \cs{psfrag}.
  \keychoice{init}{\default{true},main,false,strict,main-strict}%
    \Module{global}\Default{false}
    Determines how labels have to be initiated.  \code{false} means that
    labels are initiated when they're used the first time in the text.
    \code{true} means that labels should be initiated in the preamble with
    \cs{initcmpd}.  \code{main} is the same as \code{true} but only for main
    labels.  \code{strict} means that if an un-initiated label is used an
    error is thrown.  \code{main-strict} is the same as \code{strict} but only
    for main labels.
  \keychoice{log}{\default{true},false,silent,verbose}\Module{global}\Default{false}
    Determines how the declaration of the labels will be logged.  \code{false}
    means that no information is written to the \code{.log} file.  \code{true}
    means that basic information is written to the \code{.log} file when a
    label or a sublabel is declared.  \code{silent} is an alias of
    \code{true}.  \code{verbose} means that detailed information is written to
    the \code{.log} file when a label or a sublabel is declared.
\end{options}

\section{The Counter Settings}\label{sec:counter-settings}

\section{Formatting Labels}\label{sec:formatting-labels}

\section{Changing the Input Markers}\label{sec:chang-input-mark}

\section{Language Dependent Settings}\label{sec:lang-depend-sett}

\section{Debugging Information}\label{sec:debugg-inform}

\end{document}


